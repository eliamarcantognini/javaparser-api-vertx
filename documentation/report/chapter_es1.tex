\section{Strategia Risolutiva}

Per la realizzazione del primo punto seguendo, un approccio a task, si è deciso di mantenere lo scheletro principale realizzato nel \href{https://github.com/eliamarcantognini/pcd2122_ass1/blob/master/doc/report.pdf}{primo elaborato} (\url{https://github.com/eliamarcantognini/pcd2122_ass1/blob/master/doc/report.pdf}) andando a modificare solo quelle componenti non necessarie.

Come da specifiche la soluzione implementata è basata sui Java Executors per la gestione dei Task da completare in parallelo.
%
In particolare, si è pensato ad un singolo Task che specifica le operazioni da effettuare per l'aggiornamento delle informazioni di uno specifico corpo ad ogni iterazione.

Il modello pensato riflette l'insieme di istruzioni che nel precedente \textit{assignment} erano affidate agli agenti e per questo motivo è realizzato senza la necessità di dover restituire alcun risultato, aspetto che semplifica la gestione dei Task in quanto non esistono singoli risultati da dover ricomporre.

Seguendo questa idea viene creato un Task per ogni corpo nella simulazione, da assegnare ad un Java Executor che si occuperà della sua esecuzione.

Una volta terminata l'esecuzione di tutti i Task viene aggiornata la view in modo da rispettare il requisito di coerenza tra ciò che viene mostrato all'utente e lo stato del programma.

\section{Architettura e implementazione}
\label{section:arch_impl}

\subsection{Simulator}
\label{subs:simulator}

Simulator rispetto l'assignement precedente è stato modificato affinché implementasse un approccio a task. La parte più importante si trova dentro il metodo \texttt{exec()} il quale gestisce le iterazioni della simulazione e manda in esecuzione i task sfruttando le classi \texttt{Executors} e \texttt{ExecutorService} che abbiamo visto a lezione. \texttt{startSimulation} permette di avviare l'\texttt{exec} e sfrutta un thread differente affinché non venga bloccato il sistema.
\newline
I task vengono mandati in esecuzione sui singoli corpi e una volta avviati se ne attende il completamento.
\newline
Non è più presente una \texttt{CyclicBarrier} per la sincronizzazione dei thread al termine delle iterazioni, ma per il medesimo scopo viene utilizzato \texttt{\nameref{subs:task_sync_m}} che viene aggiornato dagli \texttt{\nameref{subs:update_task}}. Quando il monitor permette di proseguire viene aggiornato il sistema\footnote{Come nell'assignment precedente, vengono aggiornate le due liste, usate per le letture e le scritture, aumentato il tempo virtuale e aggiornata la view.} e si prosegue all'iterazione successiva.
% \newline
% Come metodo di esecuzione dell'Executor è stato scelto \texttt{newFixedThreadPool} visto a lezione.

\subsection{UpdateTask}
\label{subs:update_task}

Rappresenta il singolo modello di Task presente nel sistema e contiene le istruzioni per poter aggiornare un singolo corpo ad una iterazione.

Corrisponde quasi completamente al comportamento racchiuso all'interno degli Agenti realizzati nel primo \textit{assignment}, per cui le peculiarità sono le stesse.

Ciò che fa in più è il dover segnalare di aver completato le sue operazione al \texttt{TaskSyncMonitor} per potersi coordinare con gli altri Task.

\subsection{SyncMonitor}
\label{subs:sync_m}

Monitor utilizzato per le modifiche e controlli relativi al fatto che la simulazione possa continuare o meno. Il valore al suo interno viene modificato quando nell'interfaccia grafica viene premuto il tasto ``\texttt{STOP}" e il controllo viene effettuato da \texttt{Simulator} all'inizio di ogni nuova iterazione.

\subsection{TaskSyncMonitor}
\label{subs:task_sync_m}

Monitor usato per sincronizzare la terminazione dei task affinché si proceda alla successiva iterazione della simulazione solamente quando tutti hanno aggiornato le nuove posizioni dei corpi.
%
In particolare una volta creato l'\textit{executor} e avergli assegnato i task da completare ci si mette in attesa della loro terminazione chiamando l'istruzione \textit{wait()} sul monitor.
%
I singoli task, prima di terminare la loro esecuzione, si preoccupano di aumentare il contatore all'interno di \texttt{TaskSyncMonitor} prima della loro terminazione e, se è l'ultimo Task ad aver completato, risvegliare il Thread che era in attesa.

Al termine di ogni iterazione il contatore viene resettato per poter riutilizzare il monitor nell'iterazione successiva.


\section{Verifica delle prestazioni} \label{section:prestazioni}

Nella presente sezione si analizzeranno le prestazioni confrontando le performance di quattro modelli, i primi tre modelli già discussi nel \href{https://github.com/eliamarcantognini/pcd2122_ass1/blob/master/doc/report.pdf}{precedente elaborato} (\url{https://github.com/eliamarcantognini/pcd2122_ass1/blob/master/doc/report.pdf}) e il nuovo modello asincrono:
\begin{itemize}
    \item \textit{Sequenziale}, il modello in cui non è implementato il multithreading;
    \item \textit{1:1}, il modello in cui è implementato un mapping uno a uno tra numero di corpi e numero di thread;
    \item \textit{Fixed Thread}, il modello in cui il numero di thread è pari al numero di core utilizzabili dall'applicazione;
    \item \textit{Async}, il modello implementato mediante l'approccio a task.
\end{itemize}

L'analisi delle prestazioni è stata eseguita confrontando le performance dei quattro modelli implementati ed ognuno di loro è stata testato su un numero variabile di corpi e iterazioni.

Analizzando in dettaglio i risultati riportati nella Tabella ~\ref{tab:perf}, è possibile calcolare l'incremento della velocità di esecuzione dei vari modelli a confronto, trascurando i risultati ottenuti con il parametro \textit{nBodies} pari a 100 poiché poco significativi nell'analisi della scalabilità dell'applicazione.

Confrontando il nuovo modello \textit{Async}, si può notare come esso sia nettamente migliore dei modelli \textit{Sequenziale} e \textit{1:1} e come vada  più veloce anche del modello \textit{Fixed Thread}, con un incremento della velocità di computazione di circa il 20\%.
L'implementazione asincrona risulta quindi la più veloce nonostante abbia un andamento del tutto analogo a quella del modello \textit{Fixed Thread}. Ciò è dovuto all'implementazione degli \textit{executors} e al minor \textit{overhead}\footnote{https://www.researchgate.net/publication/2609854\_A\_Java\_ForkJoin\_Framework}\footnote{https://www.oracle.com/technical-resources/articles/java/fork-join.html} rispetto alla gestione dei thread classica.

\begin{table}[ht]
    \centering
    \begin{tabular}{|c|c|c|c|c|c|}
        \hline
        \multicolumn{2}{|c|}{Parametri} & \multicolumn{4}{|c|}{Risultati dei modelli in secondi} \\
        \hline
        N° corpi & N° iterazioni & Sequenziale & 1:1 & Fixed Thread & Async \\
        \hline
        & 1000 & 0.083 & 0.956 & 0.152 & 0.155 \\
        100 & 10000 & 0.686 & 8.874 & 0.805 & 0.860 \\
        & 50000 & 2.465 & 44.897 & 3.646 & 3.749 \\
        \hline
        & 1000 & 7.145 & 10.572 & 1.716 & 1.431 \\
        1000 & 10000 & 72.094 & 109.132 & 16.680 & 13.589 \\
        & 50000 & 359.054 & 527.608 & 83.070 & 67.392 \\
        \hline
        & 1000 & 175.565 & 58.198 & 38.800 & 32.072 \\
        5000 & 10000 & 1477.274 & 578.232 & 458.425 & 313.677 \\
        & 50000 & 9271.029 & 2897.990 & 1973.932 & 1559.505 \\
        \hline
    \end{tabular}
    \caption{Tabella riassuntiva delle misurazioni.}
    \label{tab:perf}
\end{table}

Nelle Figure ~\ref{fig:iterations} e ~\ref{fig:bodies} sono visibili i grafici dei quattro modelli messi a confronto con numeri variabili di iterazioni e corpi.

\begin{figure}[ht]
    \begin{center}
        \begin{minipage}{.5\linewidth}
            \centering
            \subfloat[]{\label{stats_iterations:a}\includegraphics[scale=.45]{img/100bodies.png}}
        \end{minipage}%
        \begin{minipage}{.5\linewidth}
            \centering
            \subfloat[]{\label{stats_iterations:b}\includegraphics[scale=.45]{img/1000bodies.png}}
        \end{minipage}\par\medskip
        \subfloat[]{\label{stats_iterations:c}\includegraphics[scale=.5]{img/5000bodies.png}}
        \centering
        \caption{Prestazioni al variare delle iterazioni}
        \label{fig:iterations}
    \end{center}
\end{figure}

\begin{figure}[ht]
    \begin{center}
        \begin{minipage}{.5\textwidth}
            \centering
            \subfloat[]{\label{stats_bodies:a}\includegraphics[scale=.45]{img/1000iterations.png}}
        \end{minipage}%
        \begin{minipage}{.5\textwidth}
            \centering
            \subfloat[]{\label{stats_bodies:b}\includegraphics[scale=.45]{img/10000iterations.png}}
        \end{minipage}\par\medskip
        \subfloat[]{\label{stats_bodies:c}\includegraphics[scale=.5]{img/50000iterations.png}}
        \centering
        \caption{Prestazioni al variare del numero di corpi}
        \label{fig:bodies}
    \end{center}    
\end{figure}

%