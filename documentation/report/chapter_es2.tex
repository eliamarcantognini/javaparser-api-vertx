\section{Analisi del problema}

Si vuole implementare una libreria che metta a disposizione una serie di funzionalità minimali per l'analisi di un progetto sviluppato utilizzando il linguaggio Java e un'iterfaccia grafica per poterla utilizzare. Tale libreria utilizzerà al suo interno un'ulteriore libreria, ovvero quella di Java Parser\footnote{Sito: \url{https://javaparser.org/} - Repository:  \url{https://github.com/javaparser/javaparser}}

\subsection{Funzionalità della libreria}
\label{subs:lib_goals}

Le funzionalità fornite dalla libreria devono dar la possibilità di:

\begin{enumerate}
    \item Analizzare un'interfaccia, restituendo il nome stesso e un elenco dei nomi dei metodi contenuti al suo interno.

    \item Analizzare una classe, restituendo il suo nome e un elenco dei metodi in essa contenuti, comprensivi di alcune informazioni ritenute essenziali: nome, modificatori (public, private, ecc.) e la loro posizione all'interno del file, specificando linea di inizio e di fine. L'analisi deve inoltre restiuire una lista dei campi dei campi, comprensiva anch'essa dei relativi modificatori.

    \item Analizzare un package, riportando il suo nome completo e un analisi delle interfacce e delle classi al suo interno.

    \item Analizzare un progetto java intero, permettendo di conoscere la sua main class e i package che lo compongono; tale analisi deve poter essere interrotta dal suo richiedente.
\end{enumerate}

Queste analisi si possono considerare a cascata, dato che le ultime si compongono di quelle precedenti.

\subsection{Funzionalità della GUI}
\label{subs:gui_goals}

La GUI deve dare la possibilità di selezionare un progetto Java e avviare un'analisi su di esso; durante quest'ultima dovranno essere mostrati gli elementi trovati all'interno del progetto. L'interfaccia grafica deve inoltre permettere di interrompere l'analisi avviata precedentemente.

\subsection{Requisiti riguardanti la metodologia di sviluppo}

Ulteriore requisito è quello di realizzare la libreria in modo asincrono, utilizzando un event-loop come architettura di riferimento facendo eseguire le relative analisi come degli handler in risposta a degli eventi.

%Si richiede anche una funzione di log della libreria che permetta di capire e verificare ciò che sta succedendo durante un'analisi in tempo reale.

% Infine, si intende realizzare una semplice applicazione provvista di GUI per poter testare l'analisi di un progetto java mostrando a schermo i risultati ottenuti oltre ai log inviati dalla libreria.
%
% L'interfaccia dovrà anche permettere di inviare un segnale alla libreria per interrompere la sua esecuzione.

\section{Strategia risolutiva}

La soluzione implementata, come specificato nei requisiti, si basa su programmazione asincrona in cui le varie operazioni sono eseguite tramite l'utilizzo di un event-loop.
%
Per mettere in pratica questo approccio si è utilizzato il framework \textit{Vertx}\footnote{\url{https://vertx.io/}}, che permette appunto di realizzare funzionalità asincrone in modo semplice, senza doversi preoccupare di molti aspetti di basso livello, come l'utilizzo effettivo dei \textit{Thread}.

In particolare, si è deciso di realizzare il comportamento legato all'analisi di una classe o di un'interfaccia come un'insieme di operazioni rispondenti ad un hanler relativo alla richiesta di ottenere informazioni relative ad, appunto quella classe o interfaccia. Potendo tale operazione poter richiedere del tempo, le istruzioni relative vengono eseguite da un ulteriore handler di Vertx, senza andare a incidere sull'event-loop\footnote{Questo per quanto concerne il calcolo; il risultato invece viene gestito dall'event-loop}.
%da considerare atomiche da eseguire su un \textit{worker thread} scelto da \textit{Vertx} tra quelli disponibili nella sua pool.
%

\redCor{Questa cosa non la metterei perché non so se al prof piace $\Rightarrow{}$} Adottando questa soluzione è avviare in parallelo l'analisi di più file, velocizzando il procedimento complessivo, soprattutto quando si chiede di eseguire l'analisi di un progetto con molti package, che a loro volta contengono molte classi o interfacce).\redCor{$\Leftarrow{}$}

La scelta di questa soluzione è anche data dal fatto che Vertx permette di ottenere da tali hanler delle \textit{Future} ottenendo quindi il risultato in modo asincrono.% tramite il meccanismo delle \textit{Future} si riesce a gestire facilmente il risultato asincrono di un'analisi, impostando semplicemente un \textit{handler} che lo gestisca coerentemente, aggiungendo all'analisi di un package per esempio.

Per quanto riguarda l'analisi di un progetto o di un package, invece, si è deciso di realizzarle in maniera differente rispetto i due casi precedenti, dato che necessitano anche di essere fermati nel caso venisse fatta richiesta, per questo si è deciso di realizzarli come \textit{Verticle} gestiti da Vertx. Queste due riutilizzano i le informazioni ottenibili grazie alle analisi delle interfacce e classi.
\redCor{Cose riguardo l'undeploy le metterei nell'implementazione.}
%
%- LO TOGLIEREI - La scelta è dovuta principalmente al fatto che a livello di analisi i report dei project e del package sono dipendenti dai report delle classi e interfacce e di conseguenza non sono considerabili come operazioni ``atomiche""\redCor{\footnote{\redCor{Non è che fa pensare che ipotizziamo i getter precedenti come atomici?}}}, perciò si è ritenuto più corretto procedere in questo modo.
%
% Inoltre, dato che uno dei requisiti è quello di rendere le operazioni della libreria interrompibili in qualsiasi momento, si è potuto sfruttare la possibilità dei \textit{Verticle} di richiamare la \textit{undeploy}\footnote{\url{https://vertx.io/docs/apidocs/io/vertx/core/Vertx.html\#undeploy-java.lang.String-}} per bloccare la sua esecuzione in modo semplice a livello di programmazione.

Siccome durante l'analisi è necessario comunicare con chi è in ascolto (in questo caso la GUI) gli elementi che vengono scoperti all'interno dei file, si è pensato di utilizzare l'\textit{event-bus} messo a disposizione da Vertx per poter inviare sia dei messaggi per indicare cosa è stato scoperto elemento è stato trovato, sia dei messaggi per eventuali errori. \redCor{Differenza tra report e "Trovato X" la metterei in implementazione}
%p sa è stato scoperto,  le quinte, inclusi eventuali errori, sia per inviare i report che man mano vengono realizzati sotto forma di json.
%

\redCor{Direi che ci scambiamo messaggi per notificare cose, ma quali siano li metterei nell'implementazione $(1)\Rightarrow{}$}
Per distinguere i vari messaggi si è ideato un semplice protocollo per cui all'inizio del messaggio viene inserita una parola chiave che permette di riconoscere il contenuto di ciò è stato inviato.
Le parole chiavi possibili sono le seguenti:

\begin{itemize}
    \item \textbf{INTERFACE\_REPORT}: indica che il contenuto successivo del messaggio è un oggetto in formato JSON e rappresenta il report di un'interfaccia.

    \item \textbf{CLASS\_REPORT}: come sopra, ma il JSON rappresenta il report di una classe.

    \item \textbf{PACKAGE\_REPORT}: come sopra, ma il JSON rappresenta il report di un package.

    \item \textbf{PROJECT\_REPORT}: come sopra, ma il JSON rappresenta il report del progetto.

    \item \textbf{METHOD\_REPORT} \label{itemize:method_report}: viene utilizzato per comunicare all'esterno il fatto che durante l'analisi di una classe o un'interfaccia è stato individuato un metodo e il resto del messaggio è un oggetto in formato JSON che contiene le informazioni su di esso che verranno poi inserite nei report.

    \item \textbf{FIELD\_REPORT}: significato simile a METHOD\_REPORT, ma riferito al campo di una classe.

    \item \textbf{ERROR}: indica che è avvenuto un errore durante l'esecuzione e il resto del messaggio è una stringa che ne indica il tipo.

    \item \textbf{INTERRUPT}: serve ad indicare all'esterno che è stato ricevuto il comando per interrompere l'esecuzione in atto.

\end{itemize}

\redCor{$\Leftarrow{}(1)$}
%INIZIO JAVAPARSER

\section{Architettura ed implementazione}

\subsection{Libreria di appoggio - Java Parser}
\label{subs:javaparser}

Libreria di appoggio utilizzata per la creazione dei metodi forniti richiesti per l'assignment, permette di creare degli \textit{Abstract Sintaxt Tree} (AST) a partire da un file \texttt{.java} e visitare i sui nodi per recuperare le informazioni di interesse.
%
Nello specifico, la visita viene effettuata tramite dei \textit{Visitor} in cui è possibile specificare il comportamento da adottare quando viene visitato un certo tipo di nodo, permettendo quindi il recupero e la memorizzazione dei dati che deve trovare la libreria, per poi restituirli al chiamante sotto forma di oggetto del tipo specificato nel \texttt{VoidVisitoAdapter}\footnote{\url{https://javadoc.io/doc/com.github.javaparser/javaparser-core/3.24.2/com/github/javaparser/ast/visitor/VoidVisitorAdapter.html}}.

\subsection{Libreria sviluppata}
\label{subs:library_developed}

Per l'implementazione della libreria si è cercato di realizzare tutte le sue componenti in modo che svolgessero il loro compito in modo asincrono.

Si è deciso di suddividere il contenuto della libreria in tre \textit{sottopackage}:

\begin{itemize}
    \item \texttt{async}: all'interno del quale sono contenute le classi direttamente coinvolte nella programmazione asincrona, che sono quindi direttamente dipendenti dal framework Vertx e dai meccanismi di questa ideologia.

    \item \texttt{visitors}: contiene le classi e le interfacce che descrivono i Visitor realizzati utilizzando Java Parser e che si occupano di estarre le informazioni contenuti nei file \texttt{.java}.

    \item \texttt{reports}: contiene le classi e interfacce in grado di contenere le informazioni relative ai report da creare durante il processo di analisi.
\end{itemize}

Tale divisione è stata fatta per dividere quelli componenti generiche che non dipendono dal paradigma ad eventi e che possono essere riutilizzate per realizzare una versione della libreria basata su un'altra metodologia di sviluppo.

Al di fuori dei package, invece, sono state estratte le due interfacce che sono utilizzate per interagire con la libreria stessa, cioè \texttt{ProjectAnalyzer} e il \texttt{Logger}.
%
La prima indica l'effettivo entry point della libreria in quanto contiene i metodi da richiamare per interagire con essa.
%
La seconda invece, come intuibile dal nome stesso, è utilizzabile per poter accedere alla lista di comandi per poter interpretare correttamente il log della libreria.

\subsubsection{Reports}

All'interno di questo p0.30

Package sono contenute le classi che descrivono i report che vengono realizzati dalla libreria e sono descritte nella figura \ref{fig:reports_interfaces}.

\begin{figure}
    \centering
    \includegraphics[width=1\linewidth]{img/ReportsInterafaces.png}
    \caption{Interfacce dei Report}
    \label{fig:reports_interfaces}
\end{figure}




I metodi getReport devono essere asincroni quindi tornano una Future del tipo relativo al report che si vuole ottenere (per esempio Future\textless ClassReport\textgreater)

\subsubsection{Visitors}
\label{subsubs:visitors}

Package con al suo interno i ``visitatori" dell'abract syntax tree generato da Java Parser; come detto in precedenza, estondono da \texttt{VoidVisitor} di Java Parser. I loro metodi vengono richiamati direttamente dal metodo \texttt{visit} della \texttt{CompilationUnit}\footnote{\url{https://javadoc.io/doc/com.github.javaparser/javaparser-core/latest/com/github/javaparser/ast/CompilationUnit.html}} di Java Parser. Il visitor base è \texttt{FileVisitor} il quale viene esteso a sua volta da \texttt{ClassesVisitor} e \texttt{InterfaceVisitor} i quali vengono richiamati in base al contesto in cui ci si trova. \redCor{Vogliamo aggiungere un'immagine?}
\redCor{Non aggiungerei altro. Al massimo cosa normalmente fa un visit.}

\subsubsection{ProjectAnalyzer}
\label{subsubs:project_analyzer}

Rappresenta l'interfaccia con i metodi che deve fornire la libreria richiesta. Tutti i metodi ritornano delle \texttt{Future} in maniera tale da poter utilizzarli in maniera asincrona rispetto al contesto in cui sta operando, eccetto l'ultimo, \texttt{analyzeProject} che permette di avviare l'analisi di un progetto. Quest'ultima comunica le informazioni che ottiene durante l'esecuzione sul topic,  passato come secondo parametro sotto forma di stringa.
\redCor{Non aggiungerei altro, semplicemente metterei un UML della classe}

\subsubsection{Logger}
\label{subsubs:logger}

\subsubsection{Async}
\label{subsubs:async}

Dentro tale package abbiamo \texttt{AsyncProjectAnalyzer} che consiste nell'implementazione di \texttt{\nameref{subsubs:project_analyzer}}. Utilizza un'istanza di Vertx per le diverse esecuzioni asincrone e il \nameref{subsubs:logger} per scambiare messaggi.

I getter relativi a interfacce e classi sono stati realizzati utilizzando il metodo \texttt{executeBlocking} di Vertx in maniera tale che la loro esecuzione non venisse effettuata sullo stesso thread dell'event-loop, il quale non deve perdere reattività. Quando il report è stato generato, viene completata la future inviandolo come parametro oltre a comunicarlo al \nameref{subsubs:logger}. Questi getter vengono utilizzati anche da \nameref{para:package_verticle} e \nameref{para:project_verticle} durante la costruzione dei relativi report.

\paragraph{ProjectVerticle}
\label{para:project_verticle}
\redCor{Non metterei molto di più del fatto che si prende diversi package verticle e che cerca il main}

\paragraph{PackageVerticle}
\label{para:package_verticle}

Classe che estende \texttt{AbstracVerticle} permette di analizzare un package. Il risulato sarà ottenibile grazie alla promise che viene passata come parametro al costruttore. Anche in questo caso è utilizzato il \nameref{subsubs:logger} per comunicare le informazioni ottenute durante l'analisi.

Quando avviene la deploy sul verticle, quest'ultimo cerca i sorgenti java all'interno del package passato come parametro al costruttore e per ognuno di loro effettua una \texttt{getInterfaceReport} o \texttt{getClassReport} relativamente al tipo di sorgente che si sta analizzando.
\newline
Il risultato del report finale verrà posto all'interno della promise solo dopo che tutte le future siano state completate\footnote{Le informazioni parziali dell'analisi saranno comunque ottenibili prima grazie al \nameref{subsubs:logger}.} grazie a \texttt{MyCompositeFuture}, un'interfaccia con soli metodi statici che permette di effettuare una \texttt{join} di una lista di \texttt{Future}.

Essendo un verticle, \nameref{para:package_verticle} può essere fermato in caso venga effettuato l'undeploy del verticle.

\redCor{A PackageVerticle non aggiungerei altro}
